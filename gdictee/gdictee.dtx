% \iffalse meta-comment
%
% gdictee.dtx
%
% This file is part of the package gdictee for use with LaTeX2e
%
% Function: Dutch dummy text from Groot Dictee (Grand Dictation)
%
% Copyright (C) 2011, 2013, 2021 Pander
%
% This program may be distributed and/or modified under the
% conditions of the LaTeX Project Public License, either version 1.2
% of this license or (at your option) any later version.
% The latest version of this license is in
%   http://www.latex-project.org/lppl.txt
% and version 1.2 or later is part of all distributions of LaTeX
% version 1999/12/01 or later.
%
% Please send error reports and suggestions for improvements to
%    Pander <pander@users.sourceforge.net>
%
%
%<*dtx>
  \ProvidesFile{gdictee.dtx}
  [2021/08/04 v0.9 Dutch dummy text from Groot Dictee (Grand Dictation)]
%</dtx>
%<package>\NeedsTeXFormat{LaTeX2e}
%<package>\ProvidesPackage{gdictee}[2021/08/04 v0.9 Dutch dummy text from Groot Dictee (Grand Dictation)]
%<driver>\ProvidesFile{gdictee.drv}
% \fi
% \GetFileInfo{gdictee.dtx}
% \CheckSum{394}
%
%% \CharacterTable
%%  {Upper-case    \A\B\C\D\E\F\G\H\I\J\K\L\M\N\O\P\Q\R\S\T\U\V\W\X\Y\Z
%%   Lower-case    \a\b\c\d\e\f\g\h\i\j\k\l\m\n\o\p\q\r\s\t\u\v\w\x\y\z
%%   Digits        \0\1\2\3\4\5\6\7\8\9
%%   Exclamation   \!     Double quote  \"     Hash (number) \#
%%   Dollar        \$     Percent       \%     Ampersand     \&
%%   Acute accent  \'     Left paren    \(     Right paren   \)
%%   Asterisk      \*     Plus          \+     Comma         \,
%%   Minus         \-     Point         \.     Solidus       \/
%%   Colon         \:     Semicolon     \;     Less than     \<
%%   Equals        \=     Greater than  \>     Question mark \?
%%   Commercial at \@     Left bracket  \[     Backslash     \\
%%   Right bracket \]     Circumflex    \^     Underscore    \_
%%   Grave accent  \`     Left brace    \{     Vertical bar  \|
%%   Right brace   \}     Tilde         \~}
%%
% \iffalse
%<*driver>
\documentclass[a4paper,11pt]{ltxdoc}
\usepackage[english]{babel}
%\usepackage[T1]{fontenc}
\usepackage{soul}
\usepackage{hyperref}
\usepackage{color}
\definecolor{dred}{rgb}{.5,0,0}
\hypersetup{colorlinks=true,urlcolor=dred}
\makeatletter
\g@addto@macro{\MacroFont}{\footnotesize}
\usepackage[DIVcalc]{typearea}
\makeatother
\begin{document}
   \DocInput{gdictee.dtx}
\end{document}
%</driver>
% \fi
% \newcommand\paket[1]{\textsf{#1}}
% \newcommand\option[1]{\textsf{#1}}
% \newcommand\name[1]{\caps{#1}}
% \title{\paket{gdictee} -- access to 72 paragraphs of Dutch
% dummy text\thanks{Version: \fileversion}}
% \author{\name{Pander}\thanks{pander@users.sourceforge.net}}
% \date{\filedate}
% \changes{v0.1}{2011/09/10}{Initial release}
% \changes{v0.2}{2011/09/11}{ASCII only and added test}
% \changes{v0.3}{2011/09/12}{Documentation, test and formatting}
% \changes{v0.4}{2011/09/12}{More hyphenation testing}
% \changes{v0.5}{2011/09/14}{Macro type, restructured and added test}
% \changes{v0.6}{2011/09/14}{C cedilla}
% \changes{v0.7}{2013/12/20}{Added texts and updated tests}
% \changes{v0.8}{2013/12/??}{Fixed ???}
% \maketitle
% \section*{Usage}
% To load the package specify
% \begin{verbatim}
%     \usepackage{gdictee}
% \end{verbatim}
% in the preamble of your document. This package provides four
% macros. \DescribeMacro{\gdictee}The most important one is
% \cmd{\gdictee}. This macro typesets Dutch paragraphs originating from the
% Grand Dictation of the Dutch Language or \emph{het Groot Dictee der
% Nederlandse Taal}. It is dummy text in contemporary Dutch provided in a
% similar way as the lipsum package is doing for Latin with \emph{Lorem Ipsum}.
%
% This dummy text can be used for optimizing layout of Dutch texts, especially
% when hyphenation is applied. Dutch has different hyphenation patterns compated
% to English or Latin but also a noticeable longer average word length. For
% this reason many hyphenation test cases are included in this package.
%
% The text is composed from dictations starting from 1990. Newer dictation will
% be added at the end to maintain backwards compatibility. Sentences of each
% dictation have been grouped somewhat equally over three paragraphs. Note that
% in 1996 and 2006 some Dutch spelling rules have changed. Nevertheless the
% texts included are on purpose not corrected for this. Only double quotation
% marks and spaced en dashes have been corrected for consequent usage.
%
% This package has an optional argument that allows one to specify the range
% of the  paragraphs. For example, \verb|\gdictee[4-57]| would typeset the paragraphs 4 to
% 57 and accordingly, \verb|\gdictee[23]| would typeset the 23\textsuperscript{rd}
% paragraph. Using \cmd{\gdictee} without its optional argument typesets
% the paragraphs specified by \DescribeMacro{\setgdicteedefault}
% \cmd{\setgdicteedefault}. This is the second macro this package
% provides. By default it is set to |1-7|, resulting in a bit more
% than one page when used with |a4paper| and the standard or the
% \KOMAScript\ classes with default font settings. To change the
% default range use \verb|\setgdicteedefault{61-63}|. Of course, the numbers
% |61| and |63| are only examples that represent the first and the last
% paragraph selected to be typeset by default.
%
% Note that the content of the optional argument is only expanded
% once. Therefore, if you want to use a macro instead of a mere number within
% the argument, make sure (most likely by using \verb|\edef|) that one
% expansion will generate a number-like string as explained above to specify
% the range of the paragraphs.
%
% If used as explained above, the paragraphs generated by \verb|\gdictee| will be
% separated by the macro \verb|\par|, or, more precisely, every paragraph will be
% terminated by \verb|\par|. Sometimes, this may cause some unintended
% effects. Therefore the package provides the option \option{nopar} that
% causes \verb|\gdictee| to omit the terminating \verb|\par|. For this purpose,
% the package should be included via
% \begin{verbatim}
%     \usepackage[nopar]{gdictee}
% \end{verbatim}
%
%
% Furthermore, a starred version of \verb|\gdictee|,
% \DescribeMacro{\gdictee*}\verb|\gdictee*|, exists that, with respect to the
% terminating \verb|\par|, does the opposite of \verb|\gdictee|: If no option
% is provided, it omits the insertion of \verb|\par| after each paragraph, if
% the option \option{nopar} is provided, it typesets the paragraphs separated by
% \verb|\par|.
%
% Note that \verb|\gdictee*| calls the macro
% \DescribeMacro{\ChangeGdicteePar} \verb|\ChangeGdicteePar| inside a
% group and subsequently calls the internal macro \verb|\@gdictee| that
% generates the output. \verb|\ChangeGdicteePar| alternates the internal macro
% \verb|\gdic@par| between
% \verb|\relax| and \verb|\par|. \verb|\gdic@par| is called at the end of each
% paragraph and therefore \verb|\ChangeGdicteePar| provides a switch to alter
% the output of this package within a single document when it is required to
% avoid additional groups.
%
% A tests can be found in gdictee-test???.tex of which you can create your own
% PDF file. The first two digits in the file name represent the font size in
% points and the third digit indicates the number of columns used. Be very
% careful when autoformatting this DTX file not to break on % hyphens which are
% sometimes part of Dutch words.
%
% \section*{Thanks}
% All paragraphs are from
% \href{http://grootdictee.nps.nl/}{http://grootdictee.nps.nl/}.
% This package is based upon package lipsum version 1.2.
% \StopEventually{}
% \clearpage
% \section*{Code}\small
%    \begin{macrocode}
%
%<*package>
\newcounter{gdic@count}
\setcounter{gdic@count}{0}

\def\gdic@par{\par}%

\DeclareOption{nopar}{\let\gdic@par\relax}
\ProcessOptions

\newcommand\gdic@default{1-7}
\newcommand\setgdicteedefault[1]{%
  \renewcommand{\gdic@default}{#1}}

\newcommand\gdic@dogdictee{%
  \ifnum\value{gdic@count}<\gdic@max\relax%
    \addtocounter{gdic@count}{1}%
    \csname gdictee@\roman{gdic@count}\endcsname%
    \gdic@dogdictee%
  \fi
}

\newcommand\gdictee{%
  \@ifstar\@@gdictee\@gdictee
}

\newcommand\ChangeGdicteePar{%
  \ifx\gdic@par\relax
    \def\gdic@par{\par}%
  \else
    \let\gdic@par\relax
  \fi
}

\newcommand\@@gdictee[1][\gdic@default]{%
  \begingroup
    \ChangeGdicteePar
    \@gdictee[#1]
  \endgroup
}
\newcommand\@gdictee[1][\gdic@default]{%
  \expandafter\gdic@minmax\expandafter{#1}%
  \setcounter{gdic@count}{\gdic@min}%
  \addtocounter{gdic@count}{-1}%
  \gdic@dogdictee%
}

\def\gdic@get#1-#2;{\def\gdic@min{#1}\def\gdic@max{#2}}
\def\gdic@stripmax#1-{\edef\gdic@max{#1}}
\def\gdic@minmax#1{%
  \gdic@get#1-\relax;%
  \edef\gdic@tmpa{\gdic@max}%
  \edef\gdic@relax{\relax}%
  \ifx\gdic@tmpa\gdic@relax\edef\gdic@max{\gdic@min}%
  \else\expandafter\gdic@stripmax\gdic@max\fi%
}

% 1990, Kees Fens, Artis (voorkeurspelling 1954), 1-3

\newcommand\gdictee@i{Onder de Noordeuropese dierentuinen komt Artis in
  anciënniteit na de fraaie dierentuin die in Londen de bezoekers in extase
  brengt. Ook Artis behoort tot die zoölogische etablissementen die door hele
  families, baby's incluis, bezocht worden. De papegaaienlaan, die naar de
  steile apenrots leidt, het jonge-mensapenhuis, het pinguïnverblijf en het oude
  berenpaleis zijn wijd en zijd gerenommeerd. De verzorging van de levende have
  is altijd geroemd. “Gij hadt het slechter kunnen treffen”, zei koning Willem
  III, helemaal uit 's-Gravenhage gekomen, in 1875 tegen de rinoceros Jan, die
  zich nochtans zeer agressief toonde.\gdic@par}

\newcommand\gdictee@ii{In het aquarium vormen de zeeëgels en de vissen uit het
  Middellandse Zeegebied het hoogtepunt; in de tropische-plantenkas is de
  kokospalm geen niemendalletje. Artis biedt helaas geen weidse uitzichten met
  gazons die noden tot neervlijen; anderzijds is de tuin ook geen labyrint vol
  bosschages. Ongewoon zijn de beelden van dinosauriërs, langs het hek
  tentoongesteld; in het midden van de tuin staat een pittoresk boeddhabeeld.
  Bezoekers die Artis sinds jaren plachten te frequenteren en die nooit ertegen
  opzagen lijdzaam in een queue te wachten, willen nog steeds koste wat kost à
  raison van een fors entreegeld in de tuin recreëren.\gdic@par}

\newcommand\gdictee@iii{De dieren huisden vroeger vaak alleen in zwaar
  getraliede, minuscule appartementen: een wildebeest, een przewalskipaard, een
  kasuaris; nu prevaleren groepsvorming en vrijheid boven het aantal soorten, al
  heeft Artis nog vele apesoorten in zijn collectie. De tuin kan geen remplaçant
  zijn van de natuur, maar het voortbestaan van de wisent, de Europese bizon, is
  wel aan Artis te danken, een bijzonderheid die appelleert aan de gevoelens van
  elke rechtgeaarde dierenvriend. Vind je dat een dierentuin de dieren in hun
  vrijheid beknot, zoals sommige actievoerende slimmeriken niet moe worden te
  stencilen, dan moet je je realiseren dat de attractieve accommodatie van de
  naoorlogse dierentuin althans veel diersoorten voor uitroeiing
  behoedt.\gdic@par}

% 1991, Kees Fens, Reizen per spoor (voorkeurspelling 1954), 4-6

\newcommand\gdictee@iv{Zullen wij, twintigste-eeuwers, het nog meemaken dat we
  per trein als geprivilegieerde burgers naar ons werk reizen, neergevlijd in de
  fauteuils van comfortabele compartimenten, nippend aan een porseleinen kop met
  koffie, voorzien van suiker en crème? Voorlopig zullen de
  tweede-klasafdelingen in de spitsuren nog wel velen tot staan dwingen; uit
  louter agressie hoorden we reeds boze tongen de cappuccino smeuïg vergelijken
  met beits of rattenkruit. Waar is de tijd van het luxueuze reizen gebleven,
  toen reizigers zich nog vermeiden in pompeuze wachtkamers met lambrizeringen
  van cipressehout en minutieus geciseleerd koperwerk, met velours beklede
  stoelen met trijpen armleuningen, en kristallijnen luchters aan het plafond?
  In deze weidse ruimten heerste een welhaast gewijde sfeer.\gdic@par}

\newcommand\gdictee@v{Men bereidde zich consciëntieus voor op een ontspannende
  reis, opgeluisterd door stedeschoon en frisse landouwen met weidend vee,
  vriendelijk struweel langs zachte glooiingen bij verzande rivierbeddinkjes,
  beneden aan het talud van de spoordijk. De geur van de stoomlocomotief
  verleende het reizen toen weliswaar een romantisch cachet, maar toch vleien
  wij ons tegenwoordig met de gedachte dat we te allen tijde zonder rook en roet
  door het landschap kunnen snellen. Veel hedendaagse reizigers zijn behept met
  de onbedwingbare begeerte per se te willen weten wat hun reisgenoten lezen. Ze
  bespioneren stiekem hun buren en zouden wensen dat er overal in de coupés
  spionnetjes hingen waarin de boek- en brochuretitels gereflecteerd
  werden.\gdic@par}

\newcommand\gdictee@vi{Aan het eind van de reis laat de adellijk ogende dame met
  haar intrigerende decolleté een flodderig niemendalletje liggen, terwijl de zo
  geëngageerd lijkende heer een minuscule thriller achterlaat die de littekens
  draagt van weinig scrupuleus gebruik. “Nee”, denk je dan, “u beider keus is
  niet de mijne.” Maar allicht brengt een ander hiertegen in dat deze gedachte
  getuigt van een hautaine ivoren-torenhouding en erop gericht is ten langen
  leste in het gevlij te komen bij het intellectuele deel van de natie. Maar
  laten we er niet lang over uitweiden: er bestaat wijd en zijd een consensus
  dat het lezen tijdens het reizen per spoor zonder malicieuze pejoratieven moet
  worden toegejuicht.\gdic@par}

% 1992, Han van Gessel, Uit eten (voorkeurspelling 1954), 7-9

\newcommand\gdictee@vii{Onze eetgewoonten blijven niet ten eeuwigen dage
  dezelfde. Het geijkte menu in menig Nederlands gezin bestond vroeger in de
  winter uit sukadelappen, hachee en savooiekool; 's zomers kwamen postelein of
  prinsessenbonen en varkensfricandeau op tafel. Het leeuwedeel van onze
  maaltijden was simpel en solide; dat men zich te goed deed in een restaurant
  was nauwelijks usance, terwijl de Vlamingen van oudsher ook buitenshuis al
  lucullisch genot kenden. Tegenwoordig kijkt niemand meer bevreemd op van een
  klein bistrootje waar omeletten en zwezeriken in een koperen kasserol worden
  opgediend. In prestigieuze etablissementen, feeëriek geïllumineerd en
  bloemrijk getooid met hyacint of chrysant, vinden zakendineetjes plaats, waar
  weinig tirannieke obers bij de foeragering soms ware Sisyfusarbeid moeten
  verrichten.\gdic@par}

\newcommand\gdictee@viii{Vroeger ontaardden zulke partijen wel eens in
  ongebreidelde orgieën; tegenwoordig vermijdt men die choquerende bacchanalen
  liever, omdat de deelnemers aan zo'n gelag na afloop als langs een liniaal
  naar huis willen rijden. Veel mensen zien rigoureus ervan af met Kerstmis zelf
  te koken, omdat zij opzien tegen het monnikenwerk in de keuken. Hoewel het
  geen aanwensel mag worden, vinden zij de kerstvakantie de periode bij uitstek
  om delicatessen in een restaurant te nuttigen, waar de gerant sliptong
  aanbeveelt of kalfsfricassee met een uitgelezen bordeaux. Specerijen als
  karwijzaad, tijm en rozemarijn, vruchten als ananassen en kiwi's alsmede
  maïskolven worden sinds lang niet meer tot de exceptionele buitenissigheden
  gerekend.\gdic@par}

\newcommand\gdictee@ix{Wie genoeg heeft van kipperagoût en toost met
  kruidenboter neemt zijn toevlucht tot de barbecue, althans als hij of zij een
  liefhebber is van gegrild vlees, overgoten met exotische sauzen. Hotellerie en
  restaurantwezen kunnen feilloos inspelen op de meest exorbitante eisen van hun
  verwende gasten, die echter soms sikkeneurig kunnen zaniken en koeioneren.
  Patrijzebout met kastanjepuree, diverse pâtés en soufflés, desserts met
  compote, confituren of ijscoupes – het vormt allemaal geen probleem. Heel in
  de verte is het gesudder van het versmade osselapje nog hoorbaar.\gdic@par}

% 1993, Han van Gessel, Het decembergevoel (voorkeurspelling 1954), 10-12

\newcommand\gdictee@x{December is sinds mensenheugenis een maand van nerveuze
  verwachting. Door de ongegeneerde activiteiten van de commercie word je de
  lang verbeide komst van Sinterklaas tegenwoordig al ruim van tevoren gewaar,
  terwijl ook de serene geneugten van het kerstfeest al vroeg tot vermoeiens
  toe worden tentoongespreid. Vroeger trad de steile gestalte van de
  filantropische bisschop met minzame nijgingen de huiskamer binnen, waar hij
  met veel egards werd ontvangen; hij werd steevast geëscorteerd door een frêle
  Zwarte Piet, die met een takkenbos van rijshout olijk dreigde slechteriken te
  kastijden.\gdic@par}

\newcommand\gdictee@xi{Kinderen, die zich eerst vermeiden met ganzenborden of
  bij het kaartspel elkaar halsstarrig de zwartepiet trachtten toe te spelen,
  plachten doodstil te worden; ze wachtten lijdzaam af of zij getrakteerd zouden
  worden op saffraangeel suikergoed, frambozerood marsepein of minuscule
  pakketjes. Op sinterklaasavond regende het dure cadeaus: portefeuilles van
  slangeleer, porseleinen eau-de-cologneflesjes, esthetisch gevormde
  bonbonniëres, cameeën met reliëfs, sjaals van shantoeng of kasjmier.
  Tegenwoordig lijkt de goedheilig man het veld te moeten ruimen ten gunste van
  de kerstman; deze schrijdt steeds vaker ten aanschouwen van velen door drukke
  winkelcentra, tenzij buiig weer dat ten enenmale verhindert.\gdic@par}

\newcommand\gdictee@xii{Deze ontwikkeling kan de doodsteek betekenen voor het
  collectieve dichterschap in dienst van het Sint-Nicolaasrijm – o droefenis!
  Wat hebben wij niet, in onze puberteit en nog lang daarna, ons best gedaan om
  een woordenbrij te creëren vol zoetgevooisde clichés en uitentreuren herhaalde
  rijmelarijen! Het feest van Kerstmis, eens allerwegen een devoot festijn met
  dennetakken en mistletoe, wordt steeds meer naar Noordamerikaans voorbeeld
  gemodelleerd – een hard gelag voor sensibele asceten en puriteinen, die nooit
  ofte nimmer willen meedoen aan te veel nieuwlichterij. Mijter en tabberd,
  hulstguirlandes en engelenhaar – zij bepalen het decembergevoel, maar hoe lang
  nog?\gdic@par}

% 1994, Han van Gessel, De dicteetor (voorkeurspelling 1954), 13-15

\newcommand\gdictee@xiii{Te midden van de diersoorten die in onze contreien met
  uitsterven worden bedreigd, neemt de dicteetor een curieuze plaats in; we
  zullen daarom de leefwijze van dit minuscule, maar agressieve insekt nader
  onder de loep nemen. Normaal leidt dit ravezwarte diertje een rustig, zo niet
  vadsig bestaan midden in zijn natuurlijke biotoop; als een volleerde meester
  in de mimicry kan het zich tegen elk vijandelijk komplot teweerstellen –
  daarmee heeft het een duidelijk prae boven zijn soortgenoten. Maar heden ten
  dage wordt het leeuwedeel van de bewoners der Lage Landen eenmaal 's jaars van
  zijn apropos gebracht, als de dicteetor plotseling haastje-repje te voorschijn
  komt en zich met veel aplomb ontpopt als een dictatoriaal
  kruidje-roer-mij-niet.\gdic@par}

\newcommand\gdictee@xiv{We zullen ervan afzien ellenlang uit te weiden over de
  hebbelijkheden van dit bijdehante beestje, maar één ding is duidelijk: doordat
  de dicteetor extreem consciëntieus is, komt hij slechts bij weinigen in het
  gevlij. Zodra de dicteetor met veel elan en bravoure de kop opsteekt, ontstaat
  onmiddellijk trammelant; met zijn vaak boude boutades bruuskeert hij
  nietsvermoedende stervelingen, want het is onloochenbaar dat tact niet zijn
  sterkste kant is. Tegen zijn choquerend gedrag is meestal geen kruid gewassen;
  woorden schieten tekort om de calamiteiten te beschrijven die hij met zijn
  vaak geëxalteerde teksten teweeg heeft gebracht.\gdic@par}

\newcommand\gdictee@xv{Maar velen houden niet van dit gejeremieer over de
  dicteetor; zij vinden dat faliekant verkeerd en applaudisseren maar al te
  graag voor zijn zegenrijk werk, dat geschraagd wordt door een ongebreidelde
  passie voor splijtende beitels, rijzende pijlers, stijgende steigers en
  vermaledijde balkenbrij – o ijselijk abracadabra! Alles tezamen ziet de wereld
  er dankzij de hocus-pocus van de dicteetor een beetje florissanter uit, maar
  of we hem ten langen leste van een tragische ondergang kunnen redden – daar
  kunnen we geen peil op trekken.\gdic@par}

% 1995, Han van Gessel, Uitzicht op zee (voorkeurspelling 1954), 16-18

\newcommand\gdictee@xvi{In deze te veel door storm en verraderlijke ijzel
  geteisterde tijden vermeit menigeen zich 's avonds in het uitzoeken van een
  attractieve vakantiebestemming. Consciëntieus bijeengeraapte toeristische
  paperassen bieden een caleidoscopisch mozaïek van stereotiepe beelden: weidse
  panorama's, onvermoede strandtaferelen, pittoreske dorpsstraatjes – de hele
  santenkraam. Velen prefereren verre, avontuurlijke reizen; de een zet zijn
  zinnen op een tournee kriskras door de woestijn van Saudi-Arabië, de ander
  kijkt reikhalzend uit naar de steilten van het Costaricaanse hooggebergte, een
  derde zoekt zijn heil te midden van de pinguïns in het antarctisch
  gebied.\gdic@par}

\newcommand\gdictee@xvii{ Nochtans stellen veel luieriken zich halsstarrig
  teweer tegen exotische uitstapjes met alle risico's van dien; zij zijn
  volmaakt gelukkig wanneer zij zich bij een temperatuur van zo'n dertig graden
  Celsius kunnen neervlijen op een zonovergoten strand, ver weg van de gevaren
  van de tseetseevlieg. In vroeger eeuwen was het strand in onze contreien
  meestentijds het domein van strandjutters, die bij nacht en ontij erop
  uittrokken; alleen als een potvis het loodje had gelegd, kwamen op sensatie
  beluste pottekijkers in groten getale erop af om het kolossale kadaver
  minutieus in ogenschouw te nemen.\gdic@par}

\newcommand\gdictee@xviii{Negentiende-eeuwse romantici ontdekten het strand als
  een fascinerende locatie; her en der ontstonden badplaatsen die met
  gepavoiseerde promenades en prestigieuze hotelaccommodaties een geneeskrachtig
  verblijf boden aan geprivilegieerde logés van goeden huize. Heden ten dage
  worden de stranden op zomerse dagen bevolkt door horden partieel of compleet
  ontblote zonaanbidders, de ravissant gebronsde lijven ingesmeerd met
  beschermende crémes; velen vinden dit plebejische schouwspel gênant en
  onoorbaar. Tengevolge van het massatoerisme heeft de strandliefhebber
  onloochenbaar concessies moeten doen; maar desondanks is er niets idyllischer
  dan een prachtige ovale kamer met wijd openslaande deuren, een majestueus
  balkon en bovenal een feeëriek uitzicht op de eindeloos deinende
  zee.\gdic@par}

% 1996, Han van Gessel, De paden op! (spelling 1995), 19-21

\newcommand\gdictee@xix{Onze verkeerswegen zijn in deze hectische naoorlogse
  tijd langzaamaan dichtgeslibd met kilometerslange files; geavanceerde
  hogesnelheidstreinen moeten hiervoor te zijner tijd enig soelaas bieden. Geen
  wonder dat Jan en alleman bij tijd en wijle snákt naar een ontspannende
  wandeling; stoïcijns en relaxed schrijdt de anders zo veelgeplaagde wandelaar
  voort langs beemd en bosschage, 's zomers bij zonneschijn alleen gekleed in
  T-shirt en kakibroek. De een wandelt graag in een dichtbegroeid bos waar
  haviken huizen en vliegezwammen welig tieren; de ander trekt met evenveel
  plezier midden door weidse weilanden, het walhalla van de groengevlekte
  graspieper en een eldorado voor dartele veulens die vrolijk hinnikend in de
  rondte galopperen.\gdic@par}

\newcommand\gdictee@xx{In de Middeleeuwen werden de kasseiwegen bevolkt door
  sjofel geklede pelgrims, met hun wijdvallende pelerines en hun
  onafscheidelijke jakobsstaf; zij prevelden hun onzevaders en overnachtten soms
  in een cisterciënzerklooster tijdens hun pelgrimage naar een ver weg gelegen
  bedevaartsoord. Begin deze eeuw trokken enthousiaste socialisten met veel elan
  erop uit, gekleed in robuuste manchesterkledij; zij werden nogal eens
  beschouwd als de risee van hun tijd, maar toch is dankzij deze
  wereldverbeteraars in spe de nieuwe mens onloochenbaar naderbij
  gekomen.\gdic@par}

\newcommand\gdictee@xxi{Heden ten dage wordt de verwoede wandelaar getrakteerd
  op een wijdvertakt netwerk van bontgekleurde paaltjes ter grootte van een
  hockeystick; ze wijzen zo nodig de weg in het labyrint van kant-en-klare
  wandeltracés, van het Noord-Groningse Pieterburen tot het Zuid-Vlaamse
  Roeselare. In steden is de culturele wandeling een rage geworden; zo kent de
  sinjorenstad Antwerpen ter wille van haar bezoekers een toeristische route
  langs parels van de Oudvlaamse gotiek, Rubens' gracieus gereconstrueerde
  woonhuis én het gerenommeerde Plantijn-museum met zijn fraaie façade en rijk
  geïllustreerde publicaties. Soms trekken de wandelaars als moderne heerscharen
  in colonnes rechttoe rechtaan over de paden; tot nu toe is het meestal goed
  toeven in Gods vrije natuur, maar wanneer maakt ook hier die vermaledijde
  filevorming een chagrijnig einde aan deze aanlokkelijke vorm van
  vrijetijdsbesteding?\gdic@par}

% 1997, Han van Gessel, De boekenwurm (spelling 1995), 22-24

\newcommand\gdictee@xxii{Heden ten dage is er nauwelijks nog plaats voor hem ten
  gevolge van alle elektronica; vader, de preses van het gezin, kijkt in een
  krankjorum hawaïhemd naar een quiz, moeder bedient laconiek de huiscomputer,
  de in T-shirts geklede kinderen manoeuvreren in de carrousel van Internet. Aan
  het eind van dit millennium dreigt naast de dicteetor en de kolibrie nog een
  buitenissig diertje te gronde te gaan: de boekenwurm; minnaars van dit vieve
  beestje luidden het afgelopen jaar, soms met te veel retoriek, de noodklok. De
  boekenwurm is een frêle schepseltje van welhaast adellijken bloede, dat zich
  graag nestelt in een met een antimakassartje beklede, pluchen fauteuil; daar
  doet het zich te goed aan een gevarieerd pakket leesvoer.\gdic@par}

\newcommand\gdictee@xxiii{Wurmen van beiderlei kunne graven zich met evenveel
  plezier een weg door avant-gardistische gedachtespinsels als door smeuïge
  pennenvruchten over slechteriken die complotten smeden, maar ten langen leste
  nul op het rekest krijgen. In de Middeleeuwen was de boekenwurm een graag
  geziene gast bij klerikale kopiisten met hun soeverein gekalligrafeerde
  manuscripten; later haalde hij daar zo nodig acrobatische capriolen uit om
  zich wiegendrukken met apocriefe teksten te kunnen toe-eigenen. Na de
  uitvinding van het drukprocédé kon het beestje naar hartelust tekeergaan; o,
  wat kon het zich vermeien in een boek – dat miraculeuze product van spirituele
  arbeid, die haarlemmerolie voor allerhande zielenroerselen, die ruggengraat
  van de cultuur!\gdic@par}

\newcommand\gdictee@xxiv{Wat zullen we uitweiden over de tijd dat menigeen van
  jongs af juist in de kerstvakantie las dat de stukken ervan afvlogen, slechts
  gestoord door het sonore geblaf van de sint-bernardshond? We eindigen daarom
  met een hartenkreet: we moeten, en dát is de portee van dit verhaal, niet gaan
  muggenziften, maar het toejuichen dat serieuze zeloten zich suf prakkiseren
  over strategieën om de boekenwurm als een feniks uit zijn as te laten
  herrijzen!\gdic@par}

% 1998, Han van Gessel, Een winteravondvertelling (spelling 1995), 25-27

\newcommand\gdictee@xxv{In deze hectische dagen tussen het sinterklaas- en het
  kerstfeest is er niets heerlijker dan onderuitgezakt te koekeloeren naar een
  smeuïge aflevering van een soapserie, het vlaggenschip van menige
  televisieomroep ter wille van de kijkcijfers. We kijken toe hoe de
  schikgodinnen de levensdraad hebben gesponnen van bon-vivants en casanova's,
  die allerwegen verwikkeld raken in zinnenprikkelende rendez-voustjes; we
  krijgen het er Spaans benauwd van als we kennisnemen van hun soms gênante
  gedachtesprongen. We zien bucolische taferelen waarin de protagonisten zich in
  een idyllisch tête-à-tête neervlijen in een classicistisch prieeltje, rondom
  omgeven door fluitenkruid en guichelheil; een herder weidt met weidse gebaren
  zijn schapen, leeuweriken zingen luid hun lied.\gdic@par}

\newcommand\gdictee@xxvi{We zwijmelen weg bij weeïge liefdesscènes, waarin
  acteurs op een berbertapijt door Cupido's pijlen getroffen gelieven
  hartstochtelijk na-apen; met bonzend hart trachten zij bij elkaar in het
  gevlij te komen om verzaligd in Morpheus' armen te kunnen neerzijgen. We
  grinniken om het debacle van een zielenpoot die na een emotioneel
  staakt-het-vuren als een freudiaanse antiheld door zijn eega sacherijnig aan
  de kant wordt geschoven; wat een kippendrift, wat een donquichotterie, wat een
  clichés!\gdic@par}

\newcommand\gdictee@xxvii{Bij tijd en wijle gaan de helden boud tekeer;
  onschuldig ogende ritjes in een janplezier kunnen ontaarden in lascieve
  bacchanalen die weinig gelijkenis vertonen met een polonaise in een
  diaconessenhuis. Zo bieden soapseries vaak stereotiepe konterfeitsels; anders
  dan de bellettrie van in marokijn gevatte boeken trakteren ze ons op
  zinnebeelden van het amoureuze leven en geven ze ons ten enenmale de kans te
  ontsnappen aan onze huis-, tuin- en keukenbesognes. Het romantisch ideaal
  scoort hoog in dit fin de siècle; toch hebben we als echte zedenprekers voor
  elk smachtend hart één stichtelijke raad: maak er in de carrousel van uw leven
  niet te veel een soapzootje van, maar beid relaxed uw tijd!\gdic@par}

% 1999, Han van Gessel, Het jachtige bestaan (spelling 1995), 28-30

\newcommand\gdictee@xxviii{De laatste kwarteeuw van dit millennium zien wij
  steeds meer veelgeplaagde mensen lijden onder de spanningen van het jachtige
  bestaan dat zij leiden; we zullen er niet over uitweiden, maar zij maken soms
  excentrieke bokkensprongen en keren niettemin meestal onverrichter zake
  huiswaarts. Vooral de vermaledijde files brengen veel hartkloppingen teweeg;
  de snelweg verandert vaak in een kafkaiaans landschap, waar machiavellisten in
  een deux-chevauxtje de hele trukendoos opentrekken om de guerrilla op de lange
  lindelaan zo nodig stiekem – pats-boem – met een watjekouw te
  beslechten.\gdic@par}

\newcommand\gdictee@xxix{Menigeen haalt in arren moede een gsm'etje
  tevoorschijn, waarmee uitentreuren met de clientèle wordt gebeld om de
  precieze locatie te melden; toeschouwers geven geen sjoege als zij midden in
  het peloton worden geconfronteerd met deze iconen van het elektronicatijdperk.
  De klok, o verafgode totem van technische productie, is voor velen de kop van
  Jut; in een onverbiddelijke cadans tiranniseert de secondewijzer met
  mathematische precisie de hele rataplan van efemere besognes.Waar is de tijd
  dat kantklossters met engelengeduld minuscule verguldsels aanbrachten op
  karmozijnen draperieën voor de soevereine chic van adellijken huize? Zij
  zetten zich voor een habbekrats aan het spinnewiel, maar maakten zich nochtans
  niet met een jantje-van-leiden ervan af.\gdic@par}

\newcommand\gdictee@xxx{In de Middeleeuwen brachten benedictijner monniken met
  hun wijdvallende pijen onder weidse kloostergewelven rust met hun gregoriaanse
  gezangen gewijd aan hun Onze-Lieve-Heer; later recupereerden velen bij
  renaissancistische melodieën van citer of klavecimbel. Heden ten dage
  praktiseren corpulente luiwammesen hun jeuïge privé-hobby's bij vlammend
  openhaardhout of barbecue; zij doen zich zonder veel egards te goed aan
  ratatouille of balkenbrij, voorzover dat geen consequenties heeft voor hun
  cholesterolgehalte.Pennenvrienden van het dictee, nu we dit fin de siècle in
  de annalen bijschrijven, is het raadzaam zo niet noodzakelijk althans even
  langzaamaan te doen; laat u niet te veel opjutten door geëxalteerde
  gedachtekronkels, maar geniet volop van de ambrozijnen verrukkelijkheden des
  levens!\gdic@par}

% 2000, Han van Gessel, De brief (spelling 1995), 31-33

\newcommand\gdictee@xxxi{Legt de handgeschreven brief te zijner tijd het loodje?
  Die vraag rijst heden ten dage bij velen te midden van de rijkgeschakeerde
  collectie goedbedoelde kerstkaarten, voorzover zij althans persoonlijke
  pennenvruchten appreciëren. De e-mailcultuur heeft op kousenvoeten een
  stilistische revolutie teweeggebracht; in bijdetijdse kattebelletjes wordt
  uitentreuren gejijd en gejoud, terwijl cedilles en i-grecs er in de brij van
  gedachtespinsels algauw allerbelabberdst van afkomen.\gdic@par}

\newcommand\gdictee@xxxii{Allerwegen wordt je voorgehouden dat het
  elektronicatijdperk de communicatiedrift rigoureus heeft aangewakkerd;
  daarover prakkiserend word je door weifelingen overvallen: wordt dankzij het
  apenstaartje niet met veel tamtam gecommuniceerd over te veel apekool? Wat
  zullen we uitweiden over dat adellijke omaatje dat opstond uit haar crapaudtje
  om, gezeten aan een ovale tafel, contact te leggen met haar geprivilegieerde
  kleindochter ver weg overzee? Haar familiale ooggetuigenverslag zat boordevol
  tante Betjes en clichés, maar niemand bracht haar van haar apropos. Bij
  zeventiende-eeuwse genreschilders was de briefschrijfster een geliefd thema;
  in sfeervol clair-obscur plachten zij een gracieus geklede vrouw af te beelden
  die de ganzenveer in een porseleinen inktpotje doopte om als een sibille de
  syllaben sierlijk te kalligraferen.\gdic@par}

\newcommand\gdictee@xxxiii{Vroeger bracht een postiljon pakketjes met
  roodgekleurde lak gedichte litanieën in een onverbiddelijke cadans op
  verafgelegen locaties; later verbeidden dorpelingen reikhalzend de lokale
  postbode, die er niet tegen opzag in weer en wind de langverwachte
  schrijfproducten – als ware hij een mecenas – in geëmailleerde brievenbussen
  neer te vlijen. Tegenwoordig moeten wij weleens grinniken om een elektronische
  liefdesbrief; wie wil nu per e-mail vleierig worden aangesproken met
  “Mijn koalaatje” of worden getrakteerd op freudiaanse minnelyriek over
  sint-janskruid en karwijzaad of andere zotteklap om in het gevlij te komen?
  Een hartenkreet ten slotte na wat zo-even te berde is gebracht: de brief, de
  wonderzalf van het leven, gaat verloren als nog slechts halsstarrig wordt
  geë-maild; blaas de schalmeien, sla de cimbalen en wek op tot een comeback van
  het epistolaire metier.\gdic@par}

% 2001, Han van Gessel en Bas van Kleef, De portemonnee, 34-36

\newcommand\gdictee@xxxiv{Allerwegen maken kassiers en caissières zich in deze
  decembermaand op voor een financieel project vanjewelste: na ommekomst van
  2001 gaan zij vol fiducie over van de nationale munt op de euro. Dit geschiedt
  ter wille van de Europese eenheid en met een soms excentriek
  euro-enthousiasme: ‘eureka’ hoorden wij zopas nog een Europagezinde
  europarlementariër op het achtuurjournaal geëxalteerd uitkrijten.\gdic@par}

\newcommand\gdictee@xxxv{Sommige eurosceptici – vooral degenen die van jongs af
  vertrouwden op de konterfeitsels in hun marokijnen portefeuille – vinden al
  die euroambtenarij algauw potjeslatijn; in geë-mailde litanieën zingen zij
  allesbehalve Brussels lof. Onder nostalgische prevelementen doen wij in een
  mêlee van somtijds veronachtzaamde klanten onze sinterklaas- en kerstaankopen
  met coupures die al decennialang rouleren; tot ons chagrijn hebben die – daar
  helpt geen lievemoederen aan – straks nog slechts antiquarische betekenis.
  Naar hartelust kopen wij cadeaus: stereotiepe barbiepoppen, cd'tjes met
  rock-'n-rollmuziek, rococoachtige kalligrafieën, emaillen kookgerei en
  geciseleerde etagères; en passant gaan wij tekeer tegen kittelorige employés
  die achter in de winkel pootaan spelen.\gdic@par}

\newcommand\gdictee@xxxvi{En à propos: wie van excursies naar Zuid-Europese
  contreien vreemde valuta overheeft, moet daar halsoverkop van af zien te
  komen; bij een volgend verblijf in den vreemde kun je er goedbeschouwd geen
  cappuccino, carpaccio of chipolatapudding meer van kopen – een krankjorume
  gedachte, maar nogal wiedes. Onze beurzen moeten een gedaanteverwisseling
  ondergaan ten gerieve van de vrijemarkteconomie; één wens zij ons hopelijk
  vergund: moge de beeltenis van de soevereine monarch, al is het geen
  staatsiefoto, nooit ofte nimmer van overheidswege worden geëlimineerd.
  Menig weifelaar zal na deze monetaire exercitie nog een tijdlang twijfelend
  blijven hoofdrekenen, maar usance zal dat zogezegd niet worden; over pakweg
  een jaar grinniken we om al dat gejeremieer over een dreigend numismatisch
  debacle in de portemonnee.\gdic@par}

% 2002, Han van Gessel en Bas van Kleef, Telefoon!, 37-39

\newcommand\gdictee@xxxvii{Peinzende treinreizigers worden heden ten dage in hun
  gecapitonneerde coupés frequent lastiggevallen; zij horen medereizigers sans
  gêne en met bravoure telefonisch tekeergaan over intieme zielenroerselen en
  particuliere aangelegenheden. Zelfs in de weidsheid van een rijke concertzaal
  wordt de ascetische klavecinist die geconcentreerd en consciëntieus een
  cantate begeleidt, rauwelijks gestoord door gedownloade riedels van een
  hinnikend paard of een hit uit de toptien; je krijgt er welhaast het
  heen-en-weer van.\gdic@par}

\newcommand\gdictee@xxxviii{Mobiel bellen is inmiddels tot in de jongste
  leeftijdscategorieën doorgedrongen; in scholen is het usance dat slimmeriken
  en apenkoppen de godganse dag sms'en om met elkaar te communiceren over
  efemere activiteiten en exotisch gedachtegoed. Het intermenselijk contact is
  rigoureus veranderd; het robuuste bakelieten telefoontoestel en de somtijds
  langverbeide conversatie zijn verdrongen door te veel geë-mailde lariekoek en
  mobiel gewauwel, waarop meestentijds geen peil valt te trekken. Nooit ofte
  nimmer had de negentiende-eeuwer Graham Bell kunnen voorzien wat zijn
  avant-gardistische inventiviteit teweeg zou brengen; het product van zijn
  visionaire exercities was indertijd prestigieus gerei van de geprivilegieerde
  bourgeoisie en puissant rijken van adellijken huize.\gdic@par}

\newcommand\gdictee@xxxix{Thans is het mobieltje een financieel interessant
  hebbedingetje voor velen; wie kan het zich ten langen leste nog permitteren op
  pad te gaan zonder dat minuscule apparaatje in zijn precieuze ovale etuitje?
  Decennialang was de telefoon een zegen; nu echter fronsen niet weinigen de
  wenkbrauwen wanneer zij kantje boord het vege lijf redden na weer eens te zijn
  gebruuskeerd door een druk pratende automobilist, die strijk-en-zet beweert
  dat hij ondanks dat frenetieke gebel veilig rijdt. Aan de onoorbare praktijken
  van gsm-junks valt nauwelijks meer te ontsnappen; nochtans lijkt dit de plaats
  om de autistische wildbeller te kapittelen: beëindig als de wiedeweerga dat
  nietsontziende gekoeioneer van de toch al veelgeplaagde medemens!\gdic@par}

% 2003, Han van Gessel en Bas van Kleef, De hittegolf, 40-42

\newcommand\gdictee@xl{In deze dagen van buiig sinterklaasweer en goeddeels
  illusoire kerstpakketten gaan de gedachten van velen terug naar de voorbije
  recordzomer; wekenlang zaten we bekaf onderuitgezakt bij onhollandse
  temperaturen, een daarbijbehorend colaatje binnen handbereik. Bij tijd en
  wijle werden we getrakteerd op ten minste eenendertig graden Celsius; sommigen
  vielen dientengevolge abrupt in katzwijm en apocalyptische tv-beelden uit New
  York illustreerden hoe gestrande metropassagiers in het pikkedonker langs
  fluorescerende noodverlichting huns weegs gingen. Niettemin wilden tallozen
  hun geplande exodus naar het zuiden voltooien; chauffeurs van beiderlei kunne
  wachtten in hun luxueuze carrosserieën in eindeloze files soms tot
  sint-juttemis, terwijl blèrende dreumesen in pyjamaatjes achter in de auto met
  hun geëtter slapstickachtige taferelen aanrichtten.\gdic@par}

\newcommand\gdictee@xli{'s Middags, nadat ten langen leste in het
  Middellandse-Zeegebied de turkooizen zee was bereikt, kwamen de minuscule
  bikinietjes tevoorschijn; alras vlijden gepiercete, half ontblote of van
  indiaanse mocassins en flatteuze tanga's voorziene lijven zich dicht opeen.
  In dierentuinen legden de gelouterde langoest, de onthechte kaketoe alsook de
  doorgaans flegmatieke kapucijneraap alledrie bijkans het loodje; op de Veluwe
  daarentegen gedijde de exotische koninginnenpage en in lommerrijke stadstuinen
  zocht de heidevlinder nijver naar nectar. Ten gevolge van de weersgesteldheid
  verspreidden zich afgelopen zomer plantensoorten die we in onze contreien
  ternauwernood kenden; ongebreideld floreerden allerwegen acacia, eucalyptus en
  apebroodboom – nog even en veldleeuweriken kunnen hun nesteldrift botvieren in
  hoog opgeschoten pampagras.\gdic@par}

\newcommand\gdictee@xlii{Majesteitelijk trok de zonnegod Helios in zijn door
  galopperende paarden voortgesleurde wagen langs het azuurblauwe zenit;
  langgeleden gebrevetteerde meteorologen bepeinsden of twijfels over rigoureuze
  klimaatschommelingen niet ijlings moesten worden gelogenstraft. Goedbeschouwd
  snakten we kortgeleden nog naar Siberische vrieskou, maar inmiddels wordt je
  door een bijdehante havo-leerling die halverwege juli nog spectaculair in de
  rondte skateboardde, alweer ontactisch toegesnauwd dat hij een zonovergoten
  pinksterdag verbeidt; het is ook nooit goed!\gdic@par}

% 2004, Jan Mulder en Remco Campert, Normen en waarden, 43-45

\newcommand\gdictee@xliii{ Ondanks zijn chronische ingewandsziekte begon de
  comicus zijn dagelijkse cadans met een eigenzinnige gedachte-exercitie over
  normen en waarden, oftewel gedragingen die hij graag geïmplanteerd zag bij
  vreemde vogels in zijn niet-geseculariseerde santenkraam. Een desperate zwaluw
  zwenkte zonder egards rakelings langs zijn rechterooglid; wat behelsde deze
  even precieze als steile duikvlucht van dit creatuur, gracieus als een
  libellenkind?\gdic@par}

\newcommand\gdictee@xliv{ De gecraqueleerde deur van zijn werkkamer ging open;
  het was drs. Mallebrootje, Tweede-Kamerlid uit Elst, tevens uitvinder van de
  geluiddemper tegen gerommel in de maag, die een exceptioneel goed humeur op 's
  mans gelaat fourneerde. Het was ijzingwekkend kil in het in Oudhollandse stijl
  gedecoreerde vertrek; het jonge ding uit de achterban gooide nog wat antraciet
  in de kachel, vooraleer zij de hooggeplaatste confrères met hun modieuze
  molières meetroonde naar de catacomben waar gebruuskeerde armlastigen bijkans
  voor pampus lagen. In deze macabere ambiance klonk in een sinistere hoek
  opeens een ridicuul gerijmel: een kwetterende kwibus sms'te met een vinkenslag
  een geïmproviseerde rap naar het mobiele nummer van de primus inter pares
  onder de pleitbezorgers voor onze normen en waarden.\gdic@par}

\newcommand\gdictee@xlv{“Verifieer en royeer, linieer en prioriteer, maar ik
  interrumpeer; ik stuur jou naar het hellevuur, jij komische karikatuur met je
  quasi-boeddhistisch surrogaat in je correlerend cosmetisch internaat.” De
  heren keken beiden uit hun ogen als paleozoën die op het internet waren
  terechtgekomen; zij deden schielijk een schietprevelementje en deleteten
  onverwijld pats-boem het vileine sms'je uit het elektronische bestand. Het
  jonge ding droeg ook een steentje bij aan de resuscitatie van de sfeer en
  friemelde wat aan haar kousenband, zodat vanonder haar plissérok een glimp van
  haar verleidelijk getatoeëerde dijtje tevoorschijn kwam; Mallebrootje
  verijsde, de comicus hield op met prakkiseren en de wereld was weer een
  heerlijke likkepot!\gdic@par}

% 2005, Herman Koch, Thuis voor de buis, 46-48

\newcommand\gdictee@xlvi{Die tintelende nazomerdag was de luie, gesubsidieerde
  bohémien zich al 's morgens vroeg te buiten gegaan aan een dolce far niente op
  zijn aftandse tweezitscrapaudtje. De antiheld deed alles op zijn elfendertigst
  en leverde de op z'n jan-boerenfluitjes in elkaar geflanste genreschilderijen
  steevast te elfder ure voor de subsidiëring af bij de mecenassen. De
  luiwammesende vrijeberoepsbeoefenaar keek naar het tv-kanaal van de op
  reclame-inkomsten gedijende supercommerciële pulpzender; daar werd net het
  eerste tête-à-tête van een twee-eiige tweeling vastgelegd.\gdic@par}

\newcommand\gdictee@xlvii{De door treiterijen geteisterde anchorman die in beeld
  kwam, was het door Jan en alleman voor overgelopen farizeeër versleten
  kijkcijferkanon; hij werd door onze kunstenmaker beschouwd als bijdetijdse
  ambassadeur van de jansaliegeest die het janhagel abusievelijk voor
  excentriciteit verslijt. “Waar blijft mijn croque-monsieur?”, blèrde de
  aquarellist naar zijn eega, een voormalige heroïneprostituee, die een half
  dozijn dictees terug, toen zij nog een frêle spring-in-'t-veld was, op de
  vip-plaats van de parkeerfaciliteit de reputatie van een promiscue
  oud-Tweede-Kamerlid te gronde had gericht.\gdic@par}

\newcommand\gdictee@xlviii{Zelf was dit op zijn eigen idee-fixe terende product
  van de bohème ook geen casanova meer; maar zoals zijn muze in een geforceerd
  up-tempo de wijnfles en de kurkentrekker binnenbracht, deed zij hem verlangen
  naar de nymfomane gepiercete lolita naast wie hij zich kortgeleden nog
  terneervlijde. Alleen de T-vormige accessoires van haar telkenmale geüpdatete
  gsm'etje verlenen haar nog een fractie sex-appeal, bedacht hij toen hij de
  etiketloze fles als een pasgeboren baby dodijnde in zijn elleboogholte. “Dat
  je als een houten klaas de reclameomzet van de nieuwe zender opvijzelt, is tot
  daar aan toe”, zei het in een feeerieke boerka gehulde ex-seksobject terwijl
  zij wat gemorste tuttifrutti onder het smyrnatapijt wegwerkte, “maar je kunt
  tenminste toch je schoenen uittrekken!”\gdic@par}

% 2006, Marin Bril, Een soapopera, 49-51

\newcommand\gdictee@xlix{De vrouw die het hippe grand café binnenliep, een miss
  twiggy aan de goede kant van de veertig, was smaakvol uitgedost in een
  bordeauxrode deux-pièces en een blouse met ovale vlakken, het precieuze
  gezicht perfect in de make-up. Op de bühne van het horeca-etablissement stond
  een diskjockey die deathmetal en francofone hiphopnummers mixte tot een
  eclectische brij; een amuzikaal karweitje waarvan nochtans al menige
  liveopname was gemaakt. Enfin.\gdic@par}

\newcommand\gdictee@l{Te midden van de pret makende jongelui leek praten ten
  enenmale onmogelijk, maar ondanks de tenhemelschreiende herrie wist de bij de
  ober in het gevlij gekomen vrouw een breezer en een schaaltje petitfours te
  bemachtigen. Ze keek welhaast smachtend naar de deur, maar werd bijwijlen
  afgeleid door een slecht afgerichte dobermannpincher die niet alleen stonk,
  maar eens te meer met continu geblaf het beatjuggelen versjteerde. Daardoor
  miste ze de entree van een fijngebouwde, wat oudere man – harrypotterbrilletje
  onder een cabareteske coupe van witte krullen – die bij de bar zijn
  stereotiepe caipirinha bestelde voordat hij haar ontwaarde.\gdic@par}

\newcommand\gdictee@li{Hij zoende haar vluchtig en begon haar op even onheuse
  als dedaigneuze wijze toe te spreken: dat gedoe van Cupido met zijn
  pijl-en-boog was over en uit; het verplichtte hem hun ontwrichte latrelatie te
  beëindigen. Zij wist zo een-twee-drie geen ad remme riposte te bedenken, maar
  verkocht hem een volmaakt kletsende oorvijg; hij belandde op de grond zodat
  hij, klem geraakt tussen het meubilair, eruitzag als een knock-out geslagen
  bokser. Hij kreunde, en terwijl de muziek uitfadede, haastte zij zich
  verdwaasd de zaak uit – onder applaus, dat wel, en toen zij de drempel
  overging en half buitenstond, was nog net te horen dat zij hem halfluid
  uitmaakte voor guppy.\gdic@par}

% 2007, Jan Wolkers, De ladder naar lust, 52-54

\newcommand\gdictee@lii{Na het verorberen van de laatste schep hete bliksem –
  een brijige stamppot, in den beginne gecreëerd door een chef-kok die ten
  eeuwigen dage het ovenvuur na aan de schenen moet worden gelegd – nam mijn
  vader zijn bijbel ter hand. “Jakobs droom te Betel, Genesis 28 vers 12. De
  jakobsladder. Jakob gaat slapen met zijn hoofd op een steen en ziet in zijn
  droom een ladder waarover engelen op en neer gaan”, zei hij onderwijzend tegen
  de kleintjes.\gdic@par}

\newcommand\gdictee@liii{“God staat boven aan de ladder en belooft hem het land
  waarop hij ligt. Daarna wordt de steen door Jakob geolied.” Mijn oudste broer
  fluisterde: “Die zevenslaper heeft zo dadelijk geen brillantine meer nodig.”
  Ik stelde me intussen voor dat ik neergevlijd, doezelend in de exquise, zurige
  sfeer van winterkost, met mijn hoofd boven op de grote kei lag die in onze
  schuur het vat afdekte waarin melkzuurbacteriën tekeergingen in fijngesneden
  wittekool. Jaren later, toen mijn vaders Schriftlezing allang was verstomd,
  zag ik de oudtestamentische scène luchthartig geïnterpreteerd terug in een
  plafondschildering in het bisschoppelijk paleis van Udine.\gdic@par}

\newcommand\gdictee@liv{In Tiepolo's Bijbelse werk ‘De droom van Jakob’
  ontwaarde ik wat ik van jongs af aan al vermoedde: dat deze gedweeë wezens
  allerminst onzijdig zijn, maar beschikken over mollige vrouwenkuiten. Bij een
  enkele vieve engel zag je zelfs, klip-en-klaar in de roze schemer van de
  opwaaiende gewaden, dat het soigneren van het schaamhaar nog niet tot het
  koninkrijk der hemelen was doorgedrongen. Het waren geen Godsgezanten, maar
  bacchanten, ladderzat op weg naar de furieuze, in het uitdeinend heelal
  weerkaatsende geluidsorgie van Jimi Hendrix, de meester van de gitaarriff, die
  ziel en zaligheid uit zijn schrepele lichaam perste.\gdic@par}

% 2008, Kristien Hemmerechts, Hartenpijn, 55-57

\newcommand\gdictee@lv{Nadat ik jarenlang zielenknijpers had geconsulteerd voor
  een aandoening die in negentiende-eeuwse traktaten als spleen wordt
  omschreven, besloot ik mijn heil niet langer in antidepressiva te zoeken. Ik
  zocht een Mariaheiligdom waar ook steilorige vroegeenentwintigste-eeuwse
  niet-gelovigen zich konden neervlijen om de genezing af te smeken van
  levercirrose, ontstoken bronchiën, X-benen of een acuut accres van genitale
  wratten.\gdic@par}

\newcommand\gdictee@lvi{Op advies van Chaucer, Groot-Brittanniës meest
  gelauwerde laatmiddeleeuwse dichter, plande ik mijn pelgrimstocht in april, de
  uitgelezen maand om naar een bedevaartsoord te trekken, of dat nu ligt in
  mediterraan of Saharaans gebied. Als fan van Hugo Claus' Oostakkerse gedichten
  lokte mij het Oost-Vlaamse Oostakker met zijn neogotische
  Onze-Lieve-Vrouwekerk, al spelen de pennenvruchten van onze betreurde bard
  veelal in een rabelaisachtig sodom en gomorra. Ik hing een schapuliermedaille
  om, speldde wat blingbling op mijn revers, leerde het Onzevader uit mijn
  hoofd, boekte overnachtingen in een x-aantal lowbudgetkloosters en sloot me
  aan bij een colonne ultradevote adellijke pelgrims.\gdic@par}

\newcommand\gdictee@lvii{Hoewel ik me had voorgenomen alleen bij het
  ecclesiastische en eschatologische stil te staan, bezweek ik voor de
  seduisante elixers, de eau des carmes en de geuzelambiek die rooms-katholieke
  monniken 's avonds serveerden. Overdag bestreed ik gênante katers en werd mijn
  queeste nu ook door een pernicieuze zondelast bezwaard, zodat ik rozenhoedjes
  mompelend voortsjokte naar mijn eindbestemming, die als een eldorado almaar
  elusiever werd. Kon een jozefshuwelijk me verlichten, wilde ik weten van een
  getonsureerde benedictijnerabt, waarop hij me in zijn gebronsde armen sloot en
  mijn larmoyante weltschmerz godzijdank eindelijk voorgoed vervloot.\gdic@par}

% 2009, Gerrit Komrij, Carrière, 58-60

\newcommand\gdictee@lviii{Het mistte zo, dat ik de afslag miste; die zin liet
  mij kennismaken met het dictee, waardoor ik stante pede van school af wilde,
  vanwege te veel voor-de-gek-houderij, al kende ik alle jaartallen van
  Krimoorlog tot Bokseropstand. Spellen was een ambigue zaak: in
  automatischepiloottoestand lukte het me vanzelf, maar zodra ik erover nadacht,
  weifelde ik of gênant zo'n fransozendakje had of niet; ja al die
  pietje-preciezerige accenten vond ik stupide, maar het gedoe met dat al of
  niet aaneenschrijven nog wel het stupiedst.\gdic@par}

\newcommand\gdictee@lix{Voortijdige schoolverlaters zijn dappere dodo's op weg
  naar Bommelskonten; maar uiteindelijk raakte ik toch nog keurig netjes
  getrouwd, met een baan onder de balkenendenorm maar wel dicht daarbij, en met
  collegaatjes die ervan uitgaan dat ik geen cultuurbarbaar ben; ik ben
  tenslotte Tweede Kamerlid. Dat fraaie traktement is meegenomen, want mijn eega
  is een funshopper; niet zo'n barbiepop die alle braderieën afstruint waar Jan
  en alleman wel een paar eurootjes kan stukslaan, maar een cliënte van pico
  bello etablissementen. Tweemaal 's weeks gaat ze een dagje statten, waarbij ze
  als een kenau in exprestempo alle jezusfreaks die de Heer en Zijn werken
  prijzen omverloopt, diverse rollatorrijpe oudjes meesleurend die aan het
  nordicwalken zijn, en dat alleen omdat ze een nieuw eau de toiletteje heeft
  gespot.\gdic@par}

\newcommand\gdictee@lx{Zwaarbeladen keert ze huiswaarts, sprenkelt eau de
  cologne uit de eau-de-colognefles en begint op internet eBay af te schuimen
  voor een zo goed als nieuw Blu-rayspelertje. Als manlief ben ik de financiële
  kop-van-jut, al kent haar koopdrift één pluspunt: jaarlijks kunnen we voor een
  extra vakantietje eropuit gaan van het geld dat onze koters met haar
  luxevoorwerpen op de vrijmarkt terugverdienen. Mijn huwelijk mag direct uit
  Dantes Hel komen, die schooldag waarop het dictee me ging tegenstaan, was wel
  het grootste debacle van mijn leven, want als ik toentertijd beter had leren
  spellen, was ik allang politicus af en nu een wat minder gedweeë
  geldschieter.\gdic@par}

% 2010, Tommy Wieringa, Kakofonie, 61-62

\newcommand\gdictee@lxi{De tijden zijn in zoverre interessant dat ze kakofonisch
  zijn; je moet bijwijlen je oren dichtstoppen tegen het tenhemelschreiende
  geblabla van de muezzins van de eeuwige vergelding in het ene oor en dat van
  de filistijnen van het geperoxideerde ressentiment in het andere. Geüpgradede
  Hoekse en Kabeljauwse twisten: de shoarmabakkers versus de aardappeleters,
  allen met vuvuzela's bewapend; hun zuurstofarme hersentjes verkleinen je
  wereld en brengen haar binnen de afrastering van hun woestijngod of hun
  karikaturale weergave van joods-christelijke axioma's en verlichtingsidealen.
  Die boeroepers nemen het licht juist weg met hun stijlmiddelen van de
  provocatie en de daarbij behorende inflatoire hyperbolen, terwijl de man in
  het midden, de verdediger van de orde, mijmert over een vreedzame
  stadssamenleving zoals bijvoorbeeld het Oost-Galicische Lemberg eens
  was.\gdic@par}

\newcommand\gdictee@lxii{Lemberg, dat multinationale en multilinguïstische
  Habsburgse babylon, die co-existentie van Asjkenazim, Roethenen, Polen,
  Duitsers, Armeniërs – door en door tolerant uit noodzaak en goede wil, laisser
  faire opgetuigd met een civilisatorische missie en een architectonisch mozaïek
  van gotiek, neoclassicisme en art deco. Nochtans werden de tijden ook
  toentertijd interessant en viel de stad achtereenvolgens toe aan het
  gerestaureerde Polen, nazi-Duitsland en de Sovjet-Unie – haar namen waren
  Lemberg, Lwów en Lviv, de Polen en Armeniërs waren verdreven en de Joden
  uitgeroeid. Hun huizen werden nu bewoond door Oekraïners; enigszins beduusd
  namen zij die verrukkelijke, lege stad in.\gdic@par}

\newcommand\gdictee@lxiii{Lemberg werd het failliet van de
  vroegtwintigste-eeuwse multiculturaliteit, alleen zijn begraafplaatsen spreken
  nog in vele talen. Wie zullen straks de halflege flatwijken van Amsterdam,
  Culemborg dan wel Gouda innemen als hun bewoners zijn verjaagd door de
  dompteurs van de thymotische woede, wier dromen de nachtmerries van de
  eerste-, tweede- en derdegeneratieallochtonen zijn? Opzichtig negeren zij die
  wet van consciëntieus bestuur die door het taoïsme wordt gepostuleerd: regeer
  de staat zoals je een klein visje bakt (behoedzaam).\gdic@par}

% 2011, Arnon Grunberg, Zelfverminking, 63-65

\newcommand\gdictee@lxiv{In dit dictee zal ik een gedeelte van Sigmund Freuds
  gedachtegoed expliqueren aan BN'ers, BV's en andere mensen die nooit ofte
  nimmer iets van hem hebben gelezen of die net als Nabokov en Van het Reve een
  diep gegronde hekel aan hem hebben. Psychoanalyticus Sigmund Freud,
  theoreticus van het oedipuscomplex, definieert in zijn essay ‘Het onbehagen
  in de cultuur’ uit 1930 cultuur als datgene waarmee wij onszelf en de wereld
  hebben geprobeerd te temmen; ook het verbod op incest is geenszins anders dan
  een verkeersregel voor de menselijke omgang.\gdic@par}

\newcommand\gdictee@lxv{Het ‘verbod op de incestueuze objectkeuze’, zoals Freud
  het accuraat formuleert, is ooit geïnitieerd om impulsief bloedvergieten te
  voorkomen; het gaat om seks, want zoons willen eigenlijk coïteren met hun
  moeder. Om dat doel te bereiken moet de zoon zijn vader, zijn concurrent,
  elimineren, want de geprivilegieerde vader wil zijn eega niet met zijn zoons
  delen – choquant en egoïstisch; het incestverbod is feitelijk een poging
  machtsverhoudingen te reguleren. In jip-en-janneketaal: wetten waardoor
  conflicten kunnen worden voorkomen; het taboe op incest blijkt niet zozeer
  een ethische kwestie, als wel een machtsaangelegenheid en hoe daarmee om te
  gaan.\gdic@par}

\newcommand\gdictee@lxvi{Nu willen jullie uiteraard wel eens weten wat dit
  allemaal met cultuur te maken heeft, welnu, cultuur wordt door Freud getypeerd
  als een soort vader die eist dat wij onze libidineuze neigingen onbevredigd
  laten in ruil voor symbolische of reële liefde. Freud noemt dit
  ‘zelfverminking’, en ook als wij ten huidigen dage de natiestaat in ogenschouw
  nemen, is er de oude ruil; de burger raakt gedomesticeerd, oftewel hij ziet af
  van de meeste driftbevrediging, een proces dat wordt geleid door politieke,
  religieuze en andere autoriteiten. Wanneer de oproerpolitie op burgers
  inslaat, zien wij, eerder dan het geweldsmonopolie van de staat in praktijk,
  enkele matig betaalde fascistoïde ambtenaren, die de gelegenheid krijgen hun
  driften te bevredigen.\gdic@par}

% 2012, Adriaan van Dis, Zijn waar wij niet zijn, 67-69

\newcommand\gdictee@lxvii{De dichter Charles Baudelaire die zonder
  kitschtrukendoos de schoonheid blasfemeerde door uit modder goud te
  destilleren, vergeleek het leven met een ziekenhuis waarin iedere patiënt
  coûte que coûte een ander bed opeist. Zo wenst de een het behaaglijk comfort
  van een vredige twee-onder-een-kapwoningwijk, de ander de kick van een
  no-goarea. Wij willen zijn, waar wij niet zijn. Scheepsarts-dichter
  Slauerhoff, die alle wereldzeeën bevoer, werd godbetert een gedweeë
  dorpsdokter en jeremieerde dat op plattelandse achterafplaatsjes never nooit
  een laag-bij-de-grondse boerenkop werd gesneld. Ervan uitgaande dat in een
  a-priorikeus ook een verlies besloten zit, wik en weeg ik mij een ons. Wat
  wordt het? Een armeluisleven zoals monniken propageren of de verrukkelijke
  weelde van de nouveaux riches?\gdic@par}

\newcommand\gdictee@lxviii{Een megasimpel ge-sms'te mededeling of een minutieus
  gekalligrafeerd epistel? Zie ik zelf oorgelabelde vaarzen in halfduistere
  ophokstallen staan, dan verlang ik stante pede naar ruziënde orang-oetangs in
  een orchideeënrijk regenwoud. Als ik spiksplinternieuwe suède molières of
  brogues moet inlopen, wil ik tegelijkertijd blootsvoets een mangrovebos in de
  afrotropische ecozone binnendringen. Spring-in-'t-veldleeuweriken in
  ruilverkavelde weiden doen mij verlangen naar de smaragdgroene valleien in
  Papoea waar een variëteit aan kasuarissen tussen de clangebieden
  weidt.\gdic@par}

\newcommand\gdictee@lxix{Deze strikt solitaire loopvogel met een basstem onder
  de drieëntwintig hertz wordt zelfs niet door oerwoudgeluidenervaren jagers
  gepercipieerd, laat staan door de sinds het indonesiseringsprogramma
  geëxcommuniceerde missionarissen. Het extreme lokt meer dan het modale, de
  imperatief meer dan het ja en amen van de hofdignitaris. Dus liever
  hermelijn dan een livrei. Maar eenmaal geëquipeerd met een peniskoker, dan
  lokt de Volendammer broek. Na de Eiffeltoren de in- en intrieste metrobuis.
  Zit ik voor een weids uitzicht mezelf te goed te doen aan de subtiel
  gearrangeerde amuses van een vijfgangendiner, dan wens ik me een patatje met
  uit de snackbar om de hoek. Deep down is het ik-besef tweeërlei: zowel
  burgemeester als swiebertje zijn.\gdic@par}

% 2013, Kees van Kooten, Een przewalskipaardenmiddel, 70-72

\newcommand\gdictee@lxx{Na koffie gedronken te hebben, begon het Groot Dictee.
  Niettegenstaande de taalcriticus Charivarius zijn macedoine ‘Is dat goed
  Nederlands?’, die verrukkelijke thesaurus vol linguïstische bêtises,
  publiceerde in 1940, zou het journaille anno hodie een raillerend exposé van
  onze pennenstrijd alsnog met dit piteuze zinnetje kunnen initiëren. Zulke
  lammenadige anastrofes vernoemde de criticaster naar zijn tante Betje, in wier
  postale verbiage een heel aantal zeugmata, polysyndetons en anakoloeten
  wiewauwde.\gdic@par}

\newcommand\gdictee@lxxi{‘Een heel aantal’ is de contaminatie van ‘een groot
  aantal’ en ‘heel veel’; vanavond irriteren wij ons aan spelfouten, mitsgaders
  aan menig grammaticaal quid pro quo – emendeer mij! Als men insinueert dat u
  heimwee heeft naar de leraar Nederlands als feniks, waarmee het zo is gegaan
  dat hij werkeloos achter de gerianiums zit, moet u niet paranoia reageren,
  aangezien die accusatie een vrijwel cliché is. Het was veelbetekend hoe elke
  media onwelwillig reflecteerde op de krankjorume tekst van het Koningslied,
  over wiens begeestering de godganse participatiemaatschappij zich
  streste.\gdic@par}

\newcommand\gdictee@lxxii{U hoort ons inziens tot een van het gardekorps
  Nederlanders die menen dat de campertiaanse schrijfwijze van het adjectief
  verrukkelijk – het megafonetische neologisme vurrukkulluk – als
  voorkeurspelling te verkiezen had geweest. Het is niet zozeer een fetisj
  voor spelling als wel het tot in de finesses breidelen van grammaticale
  valstrikken dat ons steeds overnieuw beschermt tegen een Babels imbroglio;
  mits dit verifiërende przewalskipaardenmiddel de enige wapening tegen
  nepmailtjes concretiseert. Bijgevolg beseft u zich een pseudobancaire poging
  tot phishing zorgeloos te kunnen deleten dan u leest: Er is een inactieve
  activiteit op dit rekening plaatsgevonden. Want zodan onze zuiderburen
  zeggen: Menen ligt dicht bij Kortrijk maar verre van Waregem.\gdic@par}

% 2014, Bart Chabot, Tussen niemendalletje en blankebabybilletjesprivilege, 73-80

\newcommand\gdictee@lxxiii{Geef het Dictee terug aan de kijker, kopte De
  Telegraaf vorig jaar. Daar schrok het Dictee wel even van. De genuttigde
  zwezeriken lagen plotseling zwaar op de maag. Maar na een medoc te hebben
  gedronken, toog het Dictee alsnog welgemoed aan de slag.\gdic@par}

\newcommand\gdictee@lxxiv{Dames en heren thuis en in deze parlementariërsruimte,
  bij dezen proficiat: u hebt, onder toeziend oog van koning Willy de Tweede,
  nog steeds nul fouten in uw brossel!\gdic@par}

\newcommand\gdictee@lxxv{O, als gisteren herinner ik me het eerste Dictee: na
  aankomst in een havelock met andere BN'ers bij de Eerste Kamer der
  Staten-Generaal bekroop me het rodelopergevoel. Een halfuurtje later kwam een
  kokospalm voorbij, en zee-egels uit het Middellandse Zeegebied en een kasuaris
  en nochtans; en apensoort, apenrots en apekool: een taalkundig
  houtenjassenpark, en kookte ik vanbinnen want ik kende de Van Dale niet
  vanbuiten.\gdic@par}

\newcommand\gdictee@lxxvi{De oe's en a's waren niet van de lucht tijdens dat
  gillendekeukenmeidenvertoon van het Nederlands.\gdic@par}

\newcommand\gdictee@lxxvii{Sindsdien hebben we ongelooflijk veel geleerd:
  aanwensel, bespioneren, ge-sms't en kippenragout kennen voor ons bollebozen
  geen trubbels meer, en ook uitentreuren, hawaïshirt of gestrest en een
  rock-‘n-rolllegende in goeden doen spellen wij foutloos.\gdic@par}

\newcommand\gdictee@lxxviii{Ooit mocht ik het Kinderdictee schrijven en
  vergastte de bollewangenhapsnoeten op de oeioeimachine, een perubalsempopulier
  en een tafa of West-Australische penseelstaartbuidelmuis; een gribbelgrabbel
  van woorden, alle uit de Dikke Van Dale, de toverballenautomaat van onze
  taal.\gdic@par}

\newcommand\gdictee@lxxix{Sla de Dikke willekeurig open en ontdek de
  geheimenissen van de brougham, een gesloten rijtuig voor twee personen
  getrokken door één paard; blader door die Ali Babataalschatkamer en ontdek dat
  een turbe een menigte is, en een turco een Noord-Afrikaanse inlandse
  tirailleur in Franse krijgsdienst.\gdic@par}

\newcommand\gdictee@lxxx{Dat was het jubileumdictee. Rest de vraag: wilt u de
  komende jaren meer of minder dicteeën? Het antwoord moet wel luiden: ‘Meer!
  Meer! Meer!’\gdic@par}

% 2015, Lieve Joris, Lang leve het heen-en-weer­­­, 81-86

\newcommand\gdictee@lxxxi{Ik was een pensionaatsmeisje met een goeiige nonkel
  die redemptorist was en 's zondags te allen tijde een soutane droeg. In de
  Congolese brousse praatte hij Kikongo en dronk palmwijn zo zacht als
  leguanenhuid.\gdic@par}

\newcommand\gdictee@lxxxii{Pontificaal gezeten in mijn bomma's fauteuil, onder
  de Byzantijnse afbeelding van Onze-Lieve-Vrouw van Altijddurende Bijstand in
  haar karmozijnrode gewaad, een drupje Elixir d'Anvers op het ovale
  bijzettafeltje, liet heeroom tijdens zijn congé sigarenrook de kamer in
  kringelen.\gdic@par}

\newcommand\gdictee@lxxxiii{Op mijn tweeëntwintigste verliet ik dit sacrosancte,
  breliaanse universum en verkaste naar Nederland, waar een kotelet een
  karbonade heette, caoutchouc rubber, een froufrou een pony en een brood niet
  grijs was maar bruin.\gdic@par}

\newcommand\gdictee@lxxxiv{In Mokum voelde ik me algauw senang. Ik leerde
  jij-bakken pareren, linkmiegels vermijden en ervoer mijn expatriatie nooit als
  een collocatie. Allengs maakte ik kennis met hachee, gruttenpap en
  krentjebrij, maar ook met saté en spekkoek, en at niet alleen halal maar ook
  koosjer.\gdic@par}

\newcommand\gdictee@lxxxv{‘Wordt mijn dochter daarginds niet te astrant?’,
  weifelde mijn moeder. Ze prefereerde inmiddels dat ik Neerpelts sprak – alles
  beter dan dat gutturale Hollands. Mijn vader fulmineerde tegen het perfide
  drugsbeleid van de noorderburen en hun promiscuïteitbevorderende seksshops,
  maar hun eloquentie apprecieerde hij en het Groot Dictee miste hij niet één
  keer.\gdic@par}

\newcommand\gdictee@lxxxvi{Jeminee, ben ik na al die jaren verkaasd? Vast en
  zeker, al val ik geenszins van Scylla in Charybdis wanneer ik – om te
  spreken met de onlangs verscheiden Drs. P – vice versa heen en weer vaar
  tussen beide taal- en cultuuroevers.\gdic@par}

% 2016, A.F.Th. van der Heijden, Glossolalie, 87-92

\newcommand\gdictee@lxxxvii{Geen kladderadatsch, maar wat een gebakkelei over
  het Engels als academische voertaal en de virulente impact daarvan op de
  landstaal. Hier faalt elke conciliatie middels getoost met krambamboeli.
  Complicerende factor: de impardonnabele dociliteit van de Taalunie, toch
  gesubsidieerd om het Nederlands te protegeren, niet te abhorreren.\gdic@par}

\newcommand\gdictee@lxxxviii{Men kan ook de hyperboreeërs op het perfide Albion
  ervan verdenken dat ze hun anorectische koloniale ambities willen
  revitaliseren door het idioom van de overzeese buren te infiltreren. Met
  anglicismen als boobytraps wordt alsnog een talig Angelsaksisch Gemenebest
  gecreëerd.\gdic@par}

\newcommand\gdictee@lxxxix{De vicedecaan van de bètafaculteit moge zonder
  restrictie voor verengelsing zijn, schrijver dezes is mordicus tegen. Als de
  Taalunie het Nederlands aan excavatie blootstelt, dan zal ik deze vijfde
  colonne attaqueren teneinde onze prachttaal voor dysthymie te
  behoeden.\gdic@par}

\newcommand\gdictee@xc{De voorstanders bewieroken hypocriet deze
  internationalisering, onderwijl subcutaan blakend van commercialiteit. Het
  resultaat is hooguit een hybridisch Dunglish – glossolalie zonder zweem van
  religieuze extase.\gdic@par}

\newcommand\gdictee@xci{Mijn gastheren zullen hierna wel een Groot Dictee in
  tot lingua franca geüpgraded steenkolenengels bestellen, riskerend dat
  tegenstanders van fossiele brandstoffen voor de Eerste Kamer komen
  demonstreren ten faveure van windturbines en fotovoltaïsche cellen.\gdic@par}

\newcommand\gdictee@xcii{Onze gevioleerde moerstaal kan nog qaly's winnen via
  een didactisch angehauchte opiniepeiler die alle korte ei's door lange wil
  remplaceren, daarmee zelf enigmatisch metamorfoserend tot stuttende pijler van
  een mening. Aldus kan ten langen leste de te steile helling naar een
  uitgebeende nationale stijl geslecht worden.\gdic@par}

% 2018, Wim Daniëls, Een klimaatmaat, 93-100

\newcommand\gdictee@xciii{2018 is nog niet ten einde, maar gaat, wat er tot eind
  december ook gebeurt, sowieso diverse weerrecords breken.\gdic@par}

\newcommand\gdictee@xciv{Er was deze zomer een hittegolf die lokaal naar
  verluidt 29 dagen aanhield, een on-Nederlandse toestand, en in juni werd 's
  nachts op de Veluwse vliegbasis Deelen, vlak bij Arnhem, een nachtrecord
  gemeten van 24,4 graden Celsius; zo'n temperatuur houdt je wakker, was de
  veelgehoorde klacht.\gdic@par}

\newcommand\gdictee@xcv{Uit de meteorologische trukendoos kwamen in juli tezamen
  circa 314 zonuren tevoorschijn, waartegenover een schamele 10 millimeter
  neerslag stond; de minuscule aardappels van dit jaar zijn er de wrange
  souvenirtjes van.\gdic@par}

\newcommand\gdictee@xcvi{Sommigen ervaren de toegenomen warmte als een
  cadeautje, en gaan ervan uit dat er te zijner tijd met Kerstmis al krokussen
  en lente-uitjes zullen zijn en dat carnaval enigszins subtropisch
  wordt.\gdic@par}

\newcommand\gdictee@lxcvii{Anderen voorzien een flinke toename van het aantal
  coderoodwaarschuwingen, waarvan de betekenis meestal is: hoed u voor het weer;
  ze vrezen voor een klimaatdebacle, een sterk vergrote kans op catastrofes, met
  bijvoorbeeld tekortschietende dijken als het langdurig gehoosd
  heeft.\gdic@par}

\newcommand\gdictee@xcviii{Een ongeruste weerhobbyist heeft onlangs op een
  A4'tje voor me uitgetekend, en ik geloof niet dat het nattevingerwerk was, wat
  er van de polen over zal blijven als de opwarming van de aarde doorgaat; en
  met die polen bedoelde hij niet de Polen die velen van ons kennen van hun
  schilder- en stukadoorwerk.\gdic@par}

\newcommand\gdictee@xcix{Iets van het Middellandse Zeeklimaat mag wat mij
  betreft gerust hiernaartoe komen, maar ik hoop toch ook nog weleens ‘It giet
  oan’ vanuit Friesland te mogen horen, waar ze naast het skûtsjesilen en het
  fierljeppen graag ook de Elfstedentocht in ere willen houden.\gdic@par}

\newcommand\gdictee@c{We lijken nochtans alleen nog ooit en masse naar
  Leeuwarden te kunnen afreizen voor het alom geprezen natuurijsfestijn als we
  beseffen dat het klimaat, in dit specifieke geval Koning Winter, ons alleen
  ter wille kan zijn als wij van onze kant het klimaat tegemoetkomen door een
  goede maat van het klimaat te worden, een klimaatmaat.\gdic@par}

%</package>
%    \end{macrocode}
% \Finale
\endinput

